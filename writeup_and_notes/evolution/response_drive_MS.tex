%%!TEX TS-program = latex
\documentclass[12pt,letterpaper]{article}
\bibliographystyle{evolution}
\usepackage{epsfig}                 
\usepackage[authoryear]{natbib}
\usepackage{graphicx}
\usepackage{amsmath}
\usepackage{psfrag}
\usepackage{mathabx}
\renewcommand{\baselinestretch}{1.6}
\large
\pagenumbering{arabic}
\usepackage[usenames,dvipsnames]{color}
\usepackage{fullpage}
\usepackage{multirow}
\newcommand{\ye}{\hat{y}}
\newcommand{\xe}{\hat{x}}
\usepackage{color}
\usepackage[normalem]{ulem}  
	\newcommand{\gc}[1]{{ \color{red} #1}}
	\newcommand{\yb}[1]{{ \color{blue} #1}}
\usepackage{lineno}
%\linenumbers 
\begin{document}
Dear EVOLUTION editorial team,
\\

Thank you for your consideration of our manuscript 14-0566 entitled "Sperm should evolve to make female meiosis fair." We appreciate the thorough comments by the reviewers and Associate Editor. We are submitting a new version of this manuscript in which we have incorporated reviewer comments as advised by the AE. Our revision includes versions of Models 1 and 2 with inbreeding, and has been edited to ensure clarity, and minimize the possibility of misleading our readers. We believe that this has resulted in a stronger manuscript. Below we describe in greater detail, and a point-by-point basis how we responded to specific comments of the AE and the reviewers.
\\
\\
Sincerely \\
Yaniv Brandvain and Graham Coop
\newpage


{\bf{Directive from Editor}} \\ \\
Montpellier, Nov. 6th 2014,
\\ \\
Dear Mr. Brandvain: 
I am pleased to tell you that your manuscript 14-0566 entitled "Sperm should evolve to make female meiosis fair." will be suitable for publication in Evolution pending some additional, minor revision. We appreciate your submission as well as your willingness to respond to the constructive suggestions of the reviewers and Associate Editor, Dr. Jon Wilkins.

You will find the comments from the Associate Editor and the Reviewers below. I invite you to respond to these and revise your manuscript accordingly. If your responses require additions to the text of the manuscript, please take care not to exceed our manuscript length guidelines, as these apply to the final, accepted version of the manuscript. \dots{}

Once again, thank you for submitting your manuscript to Evolution.  I look forward to the published paper.
\\  \\
Sincerely, \\
Dr. Thomas Lenormand \\
Editor, Evolution \\
thomas.lenormand@cefe.cnrs.fr

\newpage

{\bf{Comments from the Associate Editor}} -- \emph{Authors respond in italics}  \\ 


Thanks for the interesting paper. As both reviewers note, it is an interesting and important topic. I also found the biological framing compelling, and I think it should penetrate deeper into the non-theory part of the Evolution audience than many theory papers do.

Let me apologize for the slowness of the review process here. I had a heck of a time getting reviewers to agree to this one. I don't know. Maybe they somehow saw that the combined PDF was 97 pages and got scared off. Or maybe when they read the abstract and saw the umlauts on oogenesis they assumed that this was actually a mis-addressed New Yorker submission.
\\ \\
\emph{We appreciate your feedback and efforts to co\"{o}rdinate reviewers}
\\

You've got some nice reviewer comments here, which should definitely be addressed. In particular, reviewer 2, aka Gardner, made two suggestions that I see as critical.
\\ \\
{\bf{AE comment 1:}} First is the suggestion to consider inbreeding. Personally, I'm not sure it would be worth the effort at this point to write out and analyze a full-on model of a structured population with migration, etc, etc. It's a fantastic idea, and definitely worth doing, but maybe it's the next paper.

However, it would be easy to tweak each model with the inclusion of one or two F values, to give you excess homozygosity (or excess probability of green-beard matching). For example, in model 3, you mention this narrow region of parameter space that exists only when the paternal modifier functions similarly in sperm from heterozygous and homozygous fathers. One thing that inbreeding would do is increase the likelihood of occupying that space. Would that then make a sustained polymorphism biologically plausible? If so, what would that predict for the evolution of female meiosis in highly inbred populations?

{\bf{Response to AE comment 1:}}  \emph{We note that both R1 and R2 noted an interest in the case of inbreeding. We therefore have explored the behavior of model 2 [as well as model one for reference to the case of traditional drive] under six values of inbreeding. We briefly discuss these results in the modified text and make the figures [analogous to Figure 1B] in the Supplement.}
\\
\\
\\
{\bf{AE comment 2:}} Second is the taxonomic question. The phylogenetic distribution of post-fertilization completion of female meiosis is critical to the evolutionary implications of this work. Even if the data is spotty at the moment, it would be worth spelling out what is known. Plus, you could maybe make your phylogenetic hypotheses a bit more specific that way.

{\bf{Response to AE comment 2:}}  \emph{\gc{We have now included a phylogeny of animals showing the phylogenetic distribution of the stage of female meiosis, at which fertilization occurs. This has been left as a supplementary figure as we do not view this as key to our arguments; however, \yb{we believe} it is a nice addition to the paper.  \yb{We hope that this data and our pointing to it in the main text, highlights this under appreciated biology and encourages additional research}. We thank reviewers \& editor for this suggestion. }}
\\
\\
\\
{\bf{AE comment 3:}} 
A couple of comments regarding the discussion section:

In general, the discussion section is a little bit fast and loose with the evolutionary reasoning, and it is not always entirely clear what is being claimed. I get that you're trying to go for a readable voice here, to spell out the take-home message for the reader who only skimmed the results. However, I think this would benefit from thinking carefully about what the four or five key points are, and dedicating a paragraph to each.

{\bf{Response to AE comment 3:}}  \emph{We have included clear `sign-posts' throughout the discussion, and have more fully articulated the logic underlying our predictions and conclusions.}
\\
\\
\\
{\bf{AE comment 4:}} 
You mention at the end that you're not modeling female meiotic arrest. However, earlier in the discussion (ca. lines 309 \& 349) and at the end of the introduction (line 109), you make a couple of statements that border on the teleological. The current text borders on suggesting that delayed female meiosis could be favored because it permits the sperm to participate in the suppression of female meiotic drivers.

Now, that's a cool idea, but it suffers from the same limitation as evolvability arguments: it poses an adaptive benefit in terms of future selective events. It can work, but you have to be careful about the conditions. For example, either drivers would have to crop up often enough that the transient fitness reductions create a substantial fitness advantage -- or that there is a persistent fitness cost after fixation. And maybe it requires (or would benefit from) some higher-level selection.

Plus, you can flip the just-so argument around. We might expect that in species without post-fertilization completion of female meiosis, this is a consequence of female drivers that have accelerated meiosis in order to prevent interference by sperm modifiers.

Basically, there's a bunch of stuff here that would need to be explicitly modeled. So, just make sure all the appropriate caveats are in place.

{\bf{Response to AE comment 4:}}  \emph{We agree and have further clarified this point. This includes pointing out the circularity in logic of a simple version of the claim that female meiotic arrest evolves to allow sperm to combat female drivers. We do however, maintain our 3rd prediction [line 249 in our first submission] that reducing opportunities for drive could allow organisms to skip the fertilization requirement. We point out that this is a question of evolutionary maintenance rather than of evolution origin.}
\\
\\
\\
{\bf{AE comment 5:}}
More compelling in the context of the present work is the line of reasoning followed from line 354 onward, that the form of meiosis will affect the prevalence of female drivers of different sorts. I would actually like to see this expanded on a bit more.

The Om and In bit is nice, and a key piece of evidence supporting the applicability of your analysis. I do have a question regarding the predictions on crosses though. It seems that the prediction only holds to the extent that there are drivers that are in the process of being suppressed, which depends on the frequency with which new drivers arise and the rate at which they get suppressed (and presumably eventually eliminated). However, it seems like you might also expect to see the effect if there is some divergence between versions of the same drivers in different populations. Intuitively, if there is sperm-mediated suppression going on, there might be opportunities for antagonistic coevolution, with drivers selected to evade suppression (depending on the molecular mechanisms of drive and suppression, of course). This would make a similar set of predictions, I think, but would not hinge on a new driver having arisen in one of the two.

If the coevolution argument holds water, you would also expect to see faster evolutionary divergence of drive elements in species with post-fertilization meiosis. However, it might be hard to establish apples-to-apples comparisons to test that sort of thing.

{\bf{Response to AE comment 5:}} \emph{We agree with this idea, and have incorporated it into our discussion.}
\\
\\
\\
{\bf{AE minor comments:}}
\\
A couple of minor things
\\
throughout: hyphenate compound adjectives, e.g. "sperm-acting modifier" and "drive-suppressing allele"
\\
line 315: "outweighs"
\\
line 330: you forgot the gratuitous umlaut on "naive"
\\
line 372: extra space after )

{\bf{Response to AE minor comments:}} \emph{We fixed these, thanks!}

\newpage
Reviewer(s)' Comments to Author:
\\
Reviewer: 1
\\
Comments to the Author
\\Sperm favor fair female meiosis.

I enjoyed reading this interesting paper. The question is clearly one where intuition fails because of the complexity of the interacting factors and the authors? models clarify these interactions.
\\
\\
\\
{\bf{R1 comment 1:}}
Models 1-3 cover the situation in which drive in female meiosis depends on sperm (padumnal) genotype or male (paternal) genotype. The models find that the invasion of such a system of drive is unlikely because in a random-mating population drive is very rare because both sexual partners need to carry the rare allele and because drive, when it does occur, creates driving homozygotes which are assumed to have low viability and which cannot themselves undergo drive. If there is no cost to homozygotes the ?driver? should go to fixation slowly at first. Forms of inbreeding would probably also facilitate initial invasion.

{\bf{Response to R1 comment 1:}} \emph{We have added inbreeding to our analyses of models 1-2, see response to AE comment 1}
\\
\\
\\
{\bf{R1 comment 2:}}
Models 4-5 cover the situation in which an allele drives in female meiosis and this is modified by sperm genotype at the same locus (by assumption of tight linkage). Model 6 considers an unlinked sperm modifier. The interesting case in Model 4 is the suppressor allele B-. This allele initially does not suppress its own drive (because rare) but suppresses the drive of B and thus ends up in A/B- heterozygotes rather than B/B- heterozygotes. The authors say ``avoid the excess formation of low fitness homozygotes'' but I think this is slightly misleading because B and B- are different ?alleles?. B- displaces B and often comes to a lower equilibrium frequency presumably because of the effects of suppressing its own drive
(B- causes the production of higher fitness A/B- heterozygotes rather than lower fitness B-/B- homozygotes. If the difference in fitness is not too great then B- ?loses? from suppressing its own drive).

{\bf{Response to R1 comment 2:}} \emph{We have changed this section to (e.g. changing ``avoid the excess formation of low fitness homozygotes'' to ``avoid the excess formation of low fitness genotypes'') in accordance with these suggestions.}
\\
\\
\\
{\bf{R1 minor comments:}}
\\ line 272 ``so it avoid[s]'' line 274 ``the the''
\\
The take home message is that sperm alleles usually evolve to suppress female meiotic drive, whether linked (in coupling or repulsion) or unlinked to the driving locus.
\\
line 339 ``novel females meiotic drivers'' rather ``novel female meiotic drivers'' or ``novel females' meiotic drivers''
\\
line 342 ``Tthese''

{\bf{Response to R1 minor comments:}} \emph{We fixed these, thanks!}

\newpage{}

Reviewer: 2

Comments to the Author

{\bf{R2 summary:}}
This study raises an interesting question: are sperm favoured to meddle in female meiosis and, if so, what are they favoured to do? This question is motivated by the very interesting observation that, in many animals, female meiosis takes place after fertilization, giving the sperm some possible role in deciding the fate of meiotic drivers. The authors develop some mathematical models, considering cases in which sperm promote drive of matching alleles only or promote or suppress drive in general. The overall message coming from the models is that sperm do not act to promote drive, and can be favoured to suppress it.

Although I found the motivating question and background biology really interesting, I am not really convinced by the analysis.

{\bf{Response to R2 summary:}} \emph{We appreciate that the reviewer find the question, the biology, and the results compelling, and are disappointed that the reviewer is unconvinced by the analysis. However, we can find no statement in the specific comments below that suggest why the reviewer is unconvinced. We therefore respond to the comments presented below.}
\\
\\
\\
Some comments:
\\
{\bf{R2 comment 1 :}}
(1) The results are pretty obvious a priori, and don't really warrant the development and analysis of explicit models. As the authors note in the Introduction, the matching-alleles models are instances of greenbeard genes, and we already know that these are under positively frequency-dependent selection, and cannot invade from rarity in unstructured populations if there is any obligate cost involved, such as the cost of the machinery that does the allelic recognition. (This link also needs to be made in the Results and Discussion sections.) And in the scenarios where sperm are deciding the fate of drivers in an egg to which they have no genetic relatedness, it is obvious that their only concern will be to maximize the viability of the ensuing zygote, which means suppressing drive.

{\bf{Response to R2 comment 1:}} \emph{We believe that intuition without rigorous theoretical support can be dangerous. Therefore, while the results may be intuitive to R2, they were not to ourselves or to R1. We therefore believe that our results are not trivial and represent an important contribution. The complexity of our analysis [and the bit that can choke simple intuition] is that a model of drive that maintains polymorphism requires negative frequency dependance [e.g. in the form or of recessivity], and that the benefit of the greenbeard is [as R2 points out] positively frequency dependent. This makes coexistence [i.e. a joint polymorphism] unlikely. We note that the intuition of R2 with regard to our analysis appears somewhat off-base -- under some non-zero costs to a self-promoting driver, it can sweep through a population [Figure 1], a simple result that R2 does not explain with his simple intuitive assertions, above.}
\\
\\
\\
{\bf{R2 comment 2:}} (2) Explicit models would be a lot more useful if the scope of the study were extended to inbred populations, where there are more selective pressures at play and where it is less obvious to see what will be favoured. Here, driving genes would be a lot more likely to encounter identical-by-descent copies in sperm, which could act to promote collaboration on the part of the sperm (analogous to the boost that greenbeards get in structured populations, owing to increased rate of encounters). But drivers would also suffer from the reduced frequency of heterozygotes under inbreeding, and this would tend to inhibit drive. And there may be additional pressures owing to inbreeding favouring bias sex ratios, which driving sex chromosomes or, indeed, whole genomes could be important in bringing about, under female heterogamety e.g. see Gardner \& Ross (2014, ?Mating ecology explains patterns of genome elimination?, Ecol Lett, doi: 10.1111/ele.12383). An explicit model would be doing really useful work here, showing which of these effects is more important, and why. The present results for the outbreeding case, that seem a bit too trivial to stand on their own, would usefully serve to prepare the reader for more substantial results for inbreeding scenarios.

{\bf{Response to R2 comment 2:}} \emph{We agree that the case of inbreeding could be interesting. In line with comments from the AE and R1, we implement a simple model of inbreeding [see response to AE comment 1]; however, as stated by R1, completely constructing a detailed inbreeding model is beyond the scope of this ms and is best left to future work. }

\emph{As stated above, we disagree with the notion that a model which confirms the intuition of R2 is trivial. We believe that clearly and rigorously presenting novel predictions and highlighting relevant biology is a substantial contribution. We are also somewhat concerned that chasing special scenarios in the hopes of unintuitive results could district from our major and clear finding, and therefore limited our analysis of the case of inbreeding.}
\\
\\
\\
{\bf{R2 comment 3:}}  (3) I really enjoyed the biology discussed at the beginning of the Introduction. I think this could be fleshed out a lot more. For example, when you say ?in many animal species? (line 46), it would be useful to explain this more fully, perhaps with the help of a figure giving a taxonomic overview. Is post-fertilization female meiosis monophyletic, have there been reversals, are there obvious ecological or other correlates, and so on?

{\bf{Response to R2 comment 3:}}  \emph{We have now included a phylogeny of animals showing the phylogenetic distribution of the stage of female meiosis, at which fertilization occurs.}
\\


\end{document}