\documentclass[12pt,letterpaper]{article}
\usepackage{epsfig}                 
\usepackage[authoryear]{natbib}
\usepackage{graphicx}
\usepackage{amsmath}
\usepackage{psfrag}
\usepackage{mathabx}
\renewcommand{\baselinestretch}{1.6}
\large
\pagenumbering{arabic}
\usepackage{setspace}
\singlespacing
\onehalfspacing
\doublespacing
\usepackage[usenames,dvipsnames]{color}
\usepackage{fullpage}
\bibliographystyle{genetics}
\usepackage{multirow}
\newcommand{\ye}{\hat{y}}
\newcommand{\xe}{\hat{x}}
\usepackage{color}
\usepackage[normalem]{ulem}  
	\newcommand{\gc}[1]{{ \color{red} #1}}
	\newcommand{\yb}[1]{{ \color{blue} #1}}

\begin{document}
Please find attached a new manuscript for your evaluation for publication in GENETICS, titled  `Sperm should evolve to make female meiosis fair.' 
The positive and constructive reviews of our previous submission with the same title (\emph{MS ID\#: GENETICS/2014/167064})  where extremely useful in our crafting of this revision. 
Below we discuss how this paper was improved by addressing, in detail, previous concerns \emph{point-by-point}, but first we address the major  concerns with our previous submission.
\begin{itemize}
\begin{singlespace}
\item \emph{{\bf{Major concern 1 -- }} The paper as written establishes that factors expressed in sperm should generally evolve to suppress drivers of meiotic drive in female meiosis. However, as Reviewer 1 points out, this is different from showing that factors expressed in females should evolve to delay completion of meiosis until after fertilization. There is general agreement that the argument needs to be completed so that it supports all conclusions and that the justification for study of the question be improved. Reviewer 2 has provided many valuable suggestions along these lines.}  \newline \\
{\bf{Response -- }} \gc{We thank the reviewers and editor for highlighting
this point. In this revision we make it clear} that our primary
argument is that, given a chance to influence female meiosis, sperm
will evolve to increase its fairness (as reflected in our title), {\it
  and not} that females should delay the completion of meiosis so as to recruit help from sperm in making meiosis fair (as could have been interpreted from our previous draft). 
\end{singlespace}

\begin{singlespace}
\item \emph{{\bf{Major concern 2 -- }} The format of this submission departs from the style of GENETICS (model description and analysis, literature citations, etc.). As a result, it is unclear whether GENETICS is the appropriate or even the intended forum for this work.} \newline \\
{\bf{Response -- }}  We believe our manuscript presents 
(1) a rigorous and theoretical model, and 
(2) research questions that can be addressed by molecular genetics,
transmission genetics, studies of genome structure, phylogenetics and
population genetics. As such, GENETICS is a particularly appropriate
venue for our paper.  Our view is echoed by comments from Reviewer 2 who stated  `Overall I found the paper to be well written and organized, and because of this, the theory that the authors describe will be accessible to a broad spectrum of the readership of Genetics not just other theoreticians.'
The suitability of the topic of our manuscript is clear from our reference list -- a strong plurality (over 25\%) of our works cited appeared in GENETICS, placing our work squarely within the tradition of GENETICS papers.
\gc{We apologize for the formatting issues. We have corrected these
  issues, and included a short METHODS section to introduce our models.} \\
\end{singlespace} 
\end{itemize}
We thank you for your previous comments and for consideration of this revised manuscript. 
To aid in your consideration, we provide \gc{the previous reviews and a
detailed response below. }
\\
\\
Sincerely,  \\
Yaniv Brandvain and Graham Coop.

\newpage
\section*{Decision Letter}

Dear Yaniv:  \newline
%Please visit http://submit.genetics.org to view the comments of the reviewers of your submission, "Sperm should evolve to make female meiosis fair." \\

Both reviewers found the manuscript inventive and well-written.  However, both have raised substantive issues that indicate that the interesting ideas presented need more development before consideration for publication. \newline

The paper as written establishes that factors expressed in sperm should generally evolve to suppress drivers of meiotic drive in female meiosis.  However, as Reviewer 1 points out, this is different from showing that factors expressed in females should evolve to delay completion of meiosis until after fertilization.  There is general agreement that the argument needs to be completed so that it supports all conclusions and that the justification for study of the question be improved.  Reviewer 2 has provided many valuable suggestions along these lines. \newline

The format of this submission departs from the style of GENETICS (model description and analysis, literature citations, etc.).  As a result, it is unclear whether GENETICS is the appropriate or even the intended forum for this work.  I am recommending rejection of the present submission.  A manuscript that addressed all concerns raised would essentially constitute a new work, and could be reviewed as a new submission. \\ \\
Best wishes, \\
Marcy \newline 
I regret to have to bring this unwelcome news.  I hope you will continue to submit your best work for publication in the journal, and that I can bring more favorable news when you do.
%\\ 
%Sincerely,\\
%Marcy Uyenoyama\\
%Associate Editor\\
%Senior Editor
\newpage




\section*{Reviewer 1 -- Reviewer's Comments to Author \dots}

\emph{{\bf{Overview -- }} This manuscript uses simple population genetic models to evaluate the potential
for sperm haplotype (or paternal genotype) to influence female meiotic drive. 
The models are quite straightforward and the results are very clearly explained.
 In that sense, the manuscript was a pleasure to read.  This manuscript should
be published somewhere but I am not sure it is enough of a sufficient advance to
warrant publication in Genetics.}

%%I'd not use quotes right below the quoted text, seems trite.
{\bf{Response:}} \gc{We appreciate that the Reviewer enjoyed the paper.}
We do however insist that this is a sufficient advance for publication in GENETICS --
 we illustrate the expected outcomes of a previously under-appreciated feature of sperm-
 egg interactions and present a series of novel and testable hypotheses. 
\begin{enumerate}
\item
\begin{singlespace}
\emph{A major part of the motivation for the model seems to be to explain why, in
animals, meiosis is typically suspended until fertilization occurs (providing
the opportunity for males to influence female drive).  This seems a poor
motivation for the project because it would require a bizarre form of logic by
which the evolution of this delay in meiosis evolved with the ``foresight'' that
there would then be subsequent evolution of male influence on female drive. 
More directly, if the authors' interest is in the evolution of arrest in female
meiosis, then it would be better to do a model that explicitly addressed that
issue.  }

\emph{Obviously, the authors can avoid this problem by simply saying, given that
meiotic arrest occurs and provides the opportunity for males to influence drive,
we want to see what will happen.  However, it is misleading to present this
model as though it addresses the issue of the evolution of meiotic arrest.}
\end{singlespace}

\begin{singlespace}
{\bf{Response -- }} We thank the reviewer for this comment, and have reworked our manuscript to focus 
	around our major result -- that sperm are expected to evolve to prevent female meiotic drive. 
	We completely agree that this delay \gc{cannot} be favored by ``evolutionary foresight'' and have 
	modified our manuscript to ensure that readers do not
        interpret our argument in this way. \gc{We have now added
          significantly to our  revised introduction and discussion to
          clarify this point. 	
We also now more explicitly refer readers to a discussion of
        hypotheses proposed to explain female meiotic arrest (Mira
        '98), so that our discussion can be seen as being distinct
        from that. }


%	As suggested by R1 \gc{???}, we focus our discussion of the fertilization requirement to pointing out that it 
%	provides sperm an opportunity to influence female meiosis.  
%	
%	We point out that, whatever favors this arrest, the threat of sperm collaboration in meiotic drive does not 
%	represent a selective force favoring the completion of meiosis before fertilization.
%	We do argue, however, that whatever the evolutionary reason for the origin of this delay, the cost o losing 
%	sperm protection against meiotic drive could play a role in the evolutionary maintenance of the female meiotic arrest. \\
\end{singlespace}

\item
\begin{singlespace}
I did not feel that the authors made a convincing case that this is an
important phenomenon in its own right.  P. 5 has one short paragraph pointing to
what seems like some rather indirect evidence of male influence on drive.
Consequently, I find myself wondering whether this manuscript addresses a topic
of sufficient importance for Genetics.  The authors do make an interesting
prediction regarding drive effects in hybrids but in the absence of any current
evidence it is hard to know whether this is just an interesting idea that never
really happens. \\
\end{singlespace}

\begin{singlespace}
	{\bf{Response -- }}  \gc{We believe that this is an important
        phenomenon. We have strengthened our discussion of this point
        throughout the introduction,
        we have also highlighted the points raised by R2 about 
        the evidence of pervassive female drive and the opportunities for
        sperm-based products
        to interfer with female meiosis.
	We have modified the manuscript to clarify that it is quite impressive that there is evidence
	for male influence on female drive at both of the two described
        female meiotic drive elements in animals. 
	We note that although direct evidence for polymorphic female meiotic drive
        in animals is modest (i.e. these two loci in mice), 
	this likely reflects the fact that it is very hard to study
        outside of model organisms. 
Given the increasing tractibility of genetics in many traditionally
        non-model systems, and the growing awareness of the importance
        of genetic conflicts in shaping molecular systems, we think
        that it is timely to offer a framework to understand one
        aspect of sperm-egg 
        interactions and provide
strong predictions that could guide analyses.}
\end{singlespace}
\end{enumerate}

\newpage

\section*{Reviewer 2 -- Reviewer's Comments to Author \dots}

\emph{{\bf{Overview -- }} 
The focus of the manuscript entitled  ``Sperm should evolve to make female meiosis
fair'' by Yaniv Brandvain and Graham Coop is to provide a theoretical foundation
for two seemingly unrelated enigmatic observations: i) that it is common for
female meiosis to arrested at an early stage and then complete the process only
after the egg is fertilized, and ii) that there is evidence (i.e., a couple of
established cases) that the strength of female meiotic drive can be influenced
by the genotype of the sperm that fertilizes an egg.  These two seemingly
unrelated empirical patterns are functionally coupled by the authors by solving
for the conditions under which different forms of sperm-mediated modifiers of
female meiotic drive will be selectively favored.  The authors show that
``Greenbeard-effect'' modifiers that bias female meiotic drive in favor of the
allele found in the sperm evolve under a very restricted part of the total
parameter space (and rapidly fix under these conditions) while those that
prevent or reduce meiotic drive evolve under a much expanded parameter space
(and are more likely to remain polymorphic).  This asymmetry would be expected
to lead to the evolution of sperm that suppress female meiotic drive, but only
if most of the steps of female meiosis are delayed until after fertilization. So
arrested meiotic progression (until fertilization) may be an adaptation that
expands the ability of females to reduce the deleterious effects of meiotic
drive by creating the opportunity for sperm to also suppress the trait. 
	 Overall I found the paper to be well written and organized, and because of
this, the theory that the authors describe will be accessible to a broad
spectrum of the readership of Genetics -- not just other theoreticians.  Although
I have not re-derived all of the equations used in the author's models, the
special cases that I have worked through, and the logic used by the authors to
motivate the models, suggest to me that they are sound. I would like to see the
authors consider (but not necessarily follow) a few suggestions that I describe
below.  
}


{\bf{Response:}} We appreciate the reviewers comments and
encouragement. 
We address the specific comments below.


\begin{enumerate}
\begin{singlespace}
\item	\emph{The importance of the wok hinges critically on the the degree to which:  {\bf{\emph{i)}}}
female meiotic drive operates in nature, and {\bf{\emph{ii)}}} sperm influence the level of
female meiotic drive.  In species like maize, where every chromosome has two or
more established meiotic drivers (``chromosomal knobs''), the importance of
meiotic drive is obvious, but in most species there is little or no evidence for
the current operation of meiotic drive -- although low levels of transmission
ratio distortion (TRD) are becoming increasingly well established from
genome-wide SNP studies.  However, the rapid divergence of telomeric sequences
among closely related species is certainly consistent with the widespread
operation of transient episodes of meiotic drive via centromere drive.  In the
present manuscript, the prevalence and importance of female meiotic drive is
quickly summarized by the terse statements ``Meiotic drivers observed in both
plants [5, 10, 9], and animals [1, 29, 21] highlight this conflict -- the selfish
benefits and the associated pleiotropic fitness costs of drive sustain a
balanced polymorphism [24], and often generate ongoing evolutionary escalations
of drive suppressors and enhancers [8, 9].''  I think the paper would be far more
convincing and appreciated if the authors put a tad more effort into convincing
the reader that female meiotic drive is sufficiently common to be a widespread
selective force.}
\end{singlespace}
\begin{singlespace}
	{\bf{Response -- }}  We agree and have added more support of the likely importance and prevalence of female meiotic drive. 
	We hope that new theoretical work (including ours) coupled
        with the increasing easy of genetics in many system, will encourage the discovery of additional drive systems in non-model organisms in the coming years.
\end{singlespace}


\begin{singlespace}
\item	 \emph{To my knowledge, the evidence for sperm influencing the outcome of meiotic
drive is restricted to the In and Om drive systems in mice and aster transfer by
the sperm in \emph{C. elegans}.  These are clearly proof of process examples but I
think that many readers will take issue with the assumed general ability of
sperm to substantially influence the outcome of female meiotic drive. This
skepticism is to be expected  because the egg typically completes meiosis
immediately after fertilization, at a time when the sperm pronucleus is
decondensing and presumably not yet transcriptionally active. The more evidence
that the authors can provide for sperm being capable of influencing female
meiotic drive, the better this paper will be received. The authors could bolster
their case by briefly reviewing the recent evidence for extensive paternal
effects via RNA and protein packaged in the sperm, delivered to the egg, and
influencing subsequent early embryonic development. }
\end{singlespace}
\begin{singlespace}
	{\bf{Response -- }}  
	We have added a discussion of these points to our introduction. 
	Specifically, we now discuss the influence of sperm on basic processes in the egg, 
	and highlight that although data are preliminary $In$ and $Om$ are the best cases of polymorphic female drivers in animals, 
	and therefore current support for our theory is actually quite high. 
\end{singlespace}


\begin{singlespace}
\item	 \emph{Finally, female meiotic drive potentially can be suppressed by alleles
expressed in the female (traditional models) or by alleles originating in the
sperm and acting in the egg immediately after fertilization (the authors new
model).  It would be useful to compare the constraints for the evolution of both
types of suppressors to evaluate if sperm-mediated suppression of female meiosis
has some evolutionary advantage that make it easier to evolve or non-redundant
to female-coded suppressors. }
\end{singlespace}
\begin{singlespace}
	{\bf{Response -- }}  For comparison, we have added a female meiotic drive suppressor unlinked to a driver as model 
	$6^\prime$, and compared its evolution to a sperm acting driver.
	In general, the spread of an unlinked female drive suppressor is qualitatively similar to that of an unlinked sperm-acting drive suppressor; 
	however, the initial spread of the sperm suppressor is equal to (when the female suppressor is dominant), or more rapid than (otherwise) the female suppressor. 
	Given the difference in mechanism of female drive suppression in sperm as compared to females, sperm suppression of female drive is likely not functionally redundant with female drive suppression and provides a qualitatively different mechanism of suppression.
\end{singlespace}

\end{enumerate}

\subsection*{Minor Points:}
\begin{enumerate}
\begin{singlespace}
\item \emph{  On the top of page 7 the statement ``This analytical approximation, derived
from Equation (1) assuming Hardy-Weinberg'', -- precedes the introduction of
Equation (1) in the Appendix.  I suggest pointing out here that all equations
described in the text are located in the Appendix (so the reader doesn't stop
reading and start looking for them, thinking they somehow overlooked them in the
body of the manuscript). \\ \\}
{\bf{Response -- }} We can relate to the confusion and frustration of struggling to find the relevant equation. We have therefore added the parenthetical comment --   ``\emph{We note that Equation 1 and all equations discussed in the main text are presented in the appendix and derived in File S2}'' -- to the end of this paragraph. 
We also believe that the METHODS section, which we present before the results in this new manuscript, will help prepare readers for this. 
\end{singlespace}

\begin{singlespace}
\item \emph{There are 6 models described in the text but these are not described in any
detail until the Appendix.  The authors might consider placing a table in the
main body of the text that tersely describes the 6 models? \\ \\}
{\bf{Response -- }} We have added a Methods section, which introduces these six models, and makes our manuscript more in line with the GENETICS format.
\end{singlespace}

\begin{singlespace}
\item \emph{On the bottom of page 14 the authors write ``we predict that a lower
opportunity for female meiotic drive, e.g. an animal lineage with a long history
of high inbreeding or selfing, would often be accompanied by a relaxation of the
requirement of fertilization for the completion of female meiosis.''  As I
recall, species with a history if inbreeding or selfing are only at the tips of
most phylogenetic trees (i.e., they do not persist over geological time and do
not have a ``LONG history of high inbreeding or selfing'') so this prediction may
be limited. \\ \\ }
{\bf{Response -- }} We agree that biology (e.g. extinction) might conspire to make this a challenging 
prediction to test, and we point this out in our new manuscript. 
However, we believe that this hypothesis follows from our results and therefore still propose it.
\end{singlespace}

\begin{singlespace}
\item \emph{On the bottom of page 14, do you also predict that observed drive in species
hybrids will depend on the direction (which male) is used in the backcross?\\ \\}
{\bf{Response -- }}  We have expanded and clarified our prediction regarding interspecific crosses in our new manuscript. 
\end{singlespace}

\begin{singlespace}
\item 
\emph{I think that equations like $(s < d -1/2 + 3s_h/2 - ds_h)$ need parentheses to be clear. \\ \\ }
	{\bf{Response -- }}  To add clarity, we have surrounded longer in-text equations like the example above, with parentheses.
\end{singlespace}
%\yb{To add clarity, we have surrounded longer in-text equations like the example above, with parentheses.}

\end{enumerate}
\end{document}