\documentclass[12pt,letterpaper]{article}
\usepackage{epsfig}                 
\usepackage[authoryear]{natbib}
\usepackage{graphicx}
\usepackage{amsmath}
\usepackage{psfrag}
\usepackage{mathabx}
\renewcommand{\baselinestretch}{1.6}
\large
\pagenumbering{arabic}
\usepackage[usenames,dvipsnames]{color}
\usepackage{fullpage}



\begin{document} 
\begin{center}
\Large{{\bf{Text S1 of Brandvain \& Coop. \\ ``Sperm should evolve to make female meiosis fair.''}}}
\end{center}

\section*{SI Figure legends}
\subsection*{Figure S1}
Invasion analysis for a self-promoting female meiotic drive allele with
  recessive costs (selection coefficient $s$), showing the region of
  bistability. The colors, and the thin dashed contours, indicate the
  frequency the allele must be above to invade the population (note
  that these alleles reach fixation conditional on invading). In the
  white area the allele cannot invade ($x \geq1$), in the solid red
  area the allele can invade and fix when rare. In the left panel we
  show the results obtained by a grid search using the recursion, on
  the right we show the approximation obtained assuming that HWE
  holds.
  
\subsection*{Figure S2}
The unstable equilibrium frequency for a self-promoting
  female meiotic drive allele with an additive cost ($s_h=s/2$) as a
  function of the drive parameter. The solid line shows results
  obtained using the recursion, the dots our approximation given by Equation 5.

\subsection*{Figure S3}
Exact results of invasion analysis of an allele whose effect
 on female meiosis is mediated by the genotype of the fertilizing
 male.  
 A heterozygous female transmits the $B$ allele 
  with probabilities  $\frac{1}{2}$,  $d_{het}=\frac{1}{2} + h(d_{hom}-\frac{1}{2}) $ or $d_{hom}$, 
 if she mated with a n $AA$, $AB$, or $BB$ male,  respectively.  
 The allele suffers a recessive fitness cost $s$.  
 The four panels correspond to different dominance relationships.
In the parameter space below the invasion (solid) line the self-promoter
 driver can invade. In the parameter space below the fixation (long
 dash) line the self-promoter can fix. In the last
 two panels the invasion line is above the fixation line and so the
 allele can be maintained as a polymorphism in that thin slice of
 parameter space between the two lines. 
 In the final panel we show the
  fixation line (small dashes) as predicted by our HWE approximation
  ($(d_{het}-1/2)/d_{het}$) see the appendix for more details.


\section*{SI Table legends}
  \subsection*{Dataset S1}
  We detail the expected frequencies of each zygote from a given mating in Dataset S1. 
 Dataset S1 is an Excel Spreadsheet with six tabs - each one corresponding to Models 1-6. 
 Dataset S1 can be used to generate genotype frequencies after drive and selection, 
  by summing the  frequency of zygotes produced by a given mating weighted by the frequency of this mating. 
  Assuming random mating this weighting is simply the product of maternal and paternal genotype frequencies after selection in the previous generation. 


\section*{SI Appendix legends}
  \subsection*{Appendix S1}
A PDF of the Mathematica file in which we derived our analytical results and approximations.

\end{document}

