
\documentclass{article}
\usepackage{graphicx}

\begin{document}
Female meiosis is the time during development where an allele can act to directly modify its representation in the next generation. 
The potential benefits to allele what succeed in this competition and collateral damage to organismal fitness are thought to have shaped basic features of female reproduction. 
However, little attention has been directed at the odd fact that fertilization by sperm is required to complete female meiosis in many species of animal. 
At first sight this seems strange as it seems to presents a dangerous opportunity for selfish alleles in sperm to distort the outcome of female meiosis in their own favor creating the opportunity for conflicts between male and female genomes. 
In this paper, through mathematical modeling, we show that in fact quite the opposite is true, sperm are usually selected to ensure the fairness of meiosis. 
Intuitively this comes about because selfish sperm have to immediately suffer any ill consequences of their actions in the resulting zygote, and so it is strongly in sperm's interests to keep genomes from gaming female meiosis. 
We hypothesize that the wide-spread requirement of fertilization to complete female meiosis may represent a long-term collaboration between male and female gametes to ensure fairness and the production of high quality gametes.
\end{document}
