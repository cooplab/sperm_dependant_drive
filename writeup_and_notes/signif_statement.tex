
\documentclass{article}
\usepackage{graphicx}

\begin{document}
Female meiosis is the time when alleles can directly modify their transmission. Benefits to alleles that succeeds in this competition and collateral damage to organismal fitness likely shaped basic features of female reproduction. However, the observation that fertilization is often required for the completion of female meiosis appears incompatible with this view. In fact, this fertilization requirement seems to presents an opportunity for sperm-acting alleles to distort the in their own favor, creating an additional arena for conflict. We show that, counterintuitively, the opposite is true, sperm are usually selected to ensure that female meiosis is fair. Intuitively this comes about because sperm immediately suffer the company they keep in the zygote, and it is in sperm's interests to enforce fairness in female meiosis. The wide-spread requirement of fertilization to complete female meiosis may represent a long-term collaboration between sperm and mothers to ensure the production of fit offspring.
\end{document}
